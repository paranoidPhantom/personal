\documentclass[11pt,a4paper]{article}
\usepackage[utf8]{inputenc}
\usepackage[russian]{babel}
\usepackage[T2A]{fontenc}
\usepackage{geometry}
\usepackage{enumitem}
\usepackage{hyperref}

\geometry{left=2cm,right=2cm,top=2cm,bottom=2cm}

\pagestyle{empty}

\begin{document}

\begin{center}
	{\Huge \textbf{Худалла Андрей}}\\[0.3em]
	Санкт-Петербург | andrei@hudalla.dev \\
	@paranoidPhantom | github.com/paranoidPhantom
\end{center}

\section*{Образование}
\noindent\textbf{Школа №550} \hfill \textit{2022 - 2025}

\vspace{2mm}

\noindent\textbf{ИТМО} \hfill \textit{Сентябрь 2025 - н.в.} \\
\textit{Компьютерные технологии (ФИТиП)} \hfill Направление подготовки - 01.03.02 (ПМИ)

\section*{Опыт работы}
\noindent\textbf{Т-Банк} \hfill \textit{Февраль 2026 - н.в.} \\
\textit{Стажёр фронтенд-разработчик} \cdot \textit{Vue.js}

\section*{Достижения}
\begin{itemize}[leftmargin=*]
	\item Финалист + 7 место \textbf{VK Cup} (трек JS), 2023
	\item Призёр Олимпиады \textbf{PROD} (фронтенд разработка), 2024
	\item 1 место в хакатоне \textbf{Стартап за 10 часов ПИиКТ \& Тинькофф}, 2024
	\item 2 место в хакатоне \textbf{VK Mini Apps × ITMO Hack}, 2024
	\item 2 место в хакатоне \textbf{24 часа в Тинькофф}, 2024
	\item 2 место в хакатоне \textbf{Naimix code}, 2024
	\item 1 место в хакатоне \textbf{Форума инновационных центров (ФИЦ)}, 2024
	\item Призёр Олимпиады \textbf{PROD} (фронтенд разработка), 2025
\end{itemize}

\section*{Проекты}


\noindent\href{https://github.com/paranoidPhantom/tgauth}{\textbf{Telegram Auth}} \textit{TypeScript/Nuxt.js} \hfill Июнь 2024
\begin{itemize}[leftmargin=*]
	\item Библиотека авторизации через Telegram для Nuxt
\end{itemize}

\vspace{5mm}

\noindent\href{https://github.com/paranoidPhantom/eclipse}{\textbf{Telegram Bot Client}} \textit{Rust, Vue/Nuxt.js, PostgreSQL, Traefik, Protobuf} \hfill Июнь - Август 2025
\begin{itemize}[leftmargin=*]
	\item Web-клиент для взаимодействия с Telegram от лица бота; реализован двухсторонний обмен сообщениями (в том числе содержащих медиа контент).
	\item Клиент-серверное взаимодействие построено на Protobuf + WebSocket.
\end{itemize}

\section*{Скилы}

\noindent\textbf{Языки}: TypeScript, Rust, Английский (C2)

\noindent\textbf{Фреймворки}: Vue, React, Tailwind

\noindent\textbf{Инструменты}: Git, Docker, Github Actions, Neovim

\end{document}